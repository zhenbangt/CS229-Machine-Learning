\clearpage
\item \points{10} {\bf Model Calibration}

In this question we will try to understand the output $h_\theta(x)$ of the
hypothesis function of a logistic regression model, in particular why we might
treat the output as a probability (besides the fact that the sigmoid function
ensures $h_\theta(x)$ always lies in the interval $(0, 1)$).

When the probabilities outputted by a model match empirical observation, the
model is said to be \emph{well-calibrated} (or reliable). For example, if we
consider a set of examples $x^{(i)}$ for which $h_\theta(x^{(i)})  \approx
0.7$, around 70\% of those examples should have positive labels. In a
well-calibrated model, this property will hold true at every probability value.

Logistic regression tends to output well-calibrated probabilities (this is
often not true with other classifiers such as Naive Bayes, or SVMs). We will
dig a little deeper in order to understand why this is the case, and find that
the structure of the loss function explains this property.

Suppose we have a training set $\{x^{(i)},y^{(i)}\}_{i=1}^m$ with
$x^{(i)} \in \mathbb{R}^{n+1}$ and $y^{(i)} \in \{0, 1\}$. Assume we have an
intercept term $x_0^{(i)} = 1$ for all $i$. Let $\theta \in \mathbb{R}^{n+1}$
be the maximum likelihood parameters learned after training a logistic
regression model. In order for the model to be considered well-calibrated,
given any range of probabilities $(a, b)$ such that $0 \le a < b \le 1$, and
training examples $x^{(i)}$ where the model outputs $h_\theta(x^{(i)})$ fall in
the range $(a, b)$, the fraction of positives in that set of examples should be
equal to the average of the model outputs for those examples. That is, the
following property must hold:
$$
  \frac{\sum_{i\in I_{a,b}}  P\left(y^{(i)}=1|x^{(i)};\theta\right)}
       {{|\{i\in I_{a,b}\}|}}
  = \frac{\sum_{i\in I_{a,b}} \mathbb{I}\{y^{(i)} = 1\}}
         {|\{i\in I_{a,b}\}|},
$$
where $P(y=1|x;\theta) = h_\theta(x) = 1/(1+\exp(-\theta^\top x))$, $I_{a,b} =
\{ i | i \in \{1,...,m\},  h_\theta(x^{(i)}) \in (a,b)\} $ is an index set of
all training examples $x^{(i)}$ where $h_\theta(x^{(i)}) \in (a,b)$, and $|S|$
denotes the size of the set $S$.

\begin{enumerate}
  \item \subquestionpoints{5}
Show that the above property holds true for the described logistic regression
model over the range $(a,b) = (0,1)$.

\textit{Hint}: Use the fact that we include a bias term.

\ifnum\solutions=1 {
  \begin{answer}
    Firstly,

$$
	\begin{aligned}
\frac{\partial J(\theta)}{\partial \theta_j} &= -\frac{1}{m}\sum_{i=1}^m y^{(i)} (1-g(\theta^T x^{(i)}))x^{(i)}_j + (1-y^{(i)}) (-g(\theta^T x^{(i)})) x^{(i)}_j \\
    &=  -\frac{1}{m}\sum_{i=1}^m (y^{(i)}-g(\theta^T x^{(i)})) x^{(i)}_j
	\end{aligned}
$$
$$
	\begin{aligned}
\therefore	
\frac{\partial^2 J(\theta)}{\partial \theta_j \partial \theta_k} 
    &=  \frac{1}{m}\sum_{i=1}^m g(\theta^T x^{(i)})(1-g(\theta^T x^{(i)})) x^{(i)}_k x^{(i)}_j
	\end{aligned}
$$
	
And thus, for any $z$, 
$$
    \begin{aligned}
z^THz &= \sum_{j, k=0}^1 H_{jk}z_j z_k\\ 
		&= \frac{1}{m}\sum_{i=1}^m \sum_{j, k=0}^1 g(\theta^Tx^{(i)})(1 - g(\theta^Tx^{(i)}))x_j^{(i)}x_k{^{(i)}} z_j z_k\\
        &= \frac{1}{m}\sum_{i=1}^m \sum_{j, k=1}^mg(\theta^Tx^{(i)})(1 - g(\theta^Tx^{(i)}))((x^{(i)})^Tz)^2 \ge 0
    \end{aligned}
$$

Q.E.D.
\end{answer}

} \fi

  \item \subquestionpoints{3}
If we have a binary classification model that is perfectly calibrated---that
is, the property we just proved holds for any $(a, b) \subset [0, 1]$---does
this necessarily imply that the model achieves perfect accuracy? Is the
converse necessarily true? Justify your answers.

\ifnum\solutions=1 {
  \begin{answer}
    Both the answer are no. Suppose 50\% of the data is positive, then a model which always outputs $0.5$ will also be perfectly calibrated. On the other hand, a model that outputs $0.75$ for positive examples and $0.25$ for negative examples will have perfect accuracy, but is not perfectly-calibrated.
\end{answer}

} \fi

  \item \subquestionpoints{2}
Discuss what effect including $L_2$ regularization in the logistic regression
objective has on model calibration.

\ifnum\solutions=1 {
  \begin{answer}
    When a regularization $\lambda \|\theta\|$ is added, the equation in (b) becomes
    $$
    \sum_{i=1}^m y^{(i)} = \sum_{i=1}^m h_\theta(x^{(i)}) + 2\lambda \theta_0
    $$
    where $\theta_0$ is the parameter for the intercept. In general, we will not penalize this term, and in this case regularization will have no effect.

\end{answer}

} \fi

\end{enumerate}

\textbf{Remark:}  We considered the range $(a,b) = (0, 1)$. This is the only
range for which logistic regression is guaranteed to be calibrated on the
training set. When the GLM modeling assumptions hold, all ranges $(a,b) \subset
[0,1]$ are well calibrated. In addition, when the training and test set are
from the same distribution and when the model has not overfit or underfit,
logistic regression tends to be well-calibrated on unseen test data as well.
This makes logistic regression a very popular model in practice, especially
when we are interested in the level of uncertainty in the model output.
